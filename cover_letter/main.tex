%% start of file `template.tex'.
%% Copyright 2006-2013 Xavier Danaux (xdanaux@gmail.com).
%
% This work may be distributed and/or modified under the
% conditions of the LaTeX Project Public License version 1.3c,
% available at http://www.latex-project.org/lppl/.


\documentclass[12pt,a4paper,sans]{moderncv}        % possible options include font size ('10pt', '11pt' and '12pt'), paper size ('a4paper', 'letterpaper', 'a5paper', 'legalpaper', 'executivepaper' and 'landscape') and font family ('sans' and 'roman')

% moderncv themes
\moderncvstyle{classic}                           % style options are 'casual' (default), 'classic', 'oldstyle' and 'banking'
\moderncvcolor{orange}                               % color options 'blue' (default), 'orange', 'green', 'red', 'purple', 'grey' and 'black'
%\renewcommand{\familydefault}{\sfdefault}         % to set the default font; use '\sfdefault' for the default sans serif font, '\rmdefault' for the default roman one, or any tex font name
%\nopagenumbers{}                                  % uncomment to suppress automatic page numbering for CVs longer than one page

% character encoding
\usepackage[utf8]{inputenc}                       % if you are not using xelatex ou lualatex, replace by the encoding you are using
%\usepackage{CJKutf8}                              % if you need to use CJK to typeset your resume in Chinese, Japanese or Korean

% adjust the page margins
\usepackage[scale=0.77]{geometry}
\setlength{\hintscolumnwidth}{5cm}                % if you want to change the width of the column with the dates
%\setlength{\makecvtitlenamewidth}{10cm}           % for the 'classic' style, if you want to force the width allocated to your name and avoid line breaks. be careful though, the length is normally calculated to avoid any overlap with your personal info; use this at your own typographical risks...

% personal data
\name{Davide}{Spataro}
%\title{Resumé title}                               % optional, remove / comment the line if not wanted
\address{Via P. Pietro Bucci}{Rende, 87136}{Italy}% optional, remove / comment the line if not wanted; the "postcode city" and and "country" arguments can be omitted or provided empty
\phone[mobile]{+39~(392)~65~800~15}                   % optional, remove / comment the line if not wanted
\phone[mobile]{+44~(0795)~86~793~367}                         % optional, remove / comment the line if not wanted
\phone[fixed]{+39~(0963)~8860~97}                      % optional, remove / comment the line if not wanted
\email{davide90.spataro@gmail.com}
%\email{davidespataro@davidespataro.com}                               % optional, remove / comment the line if not wanted
\homepage{www.davidespataro.it}                         % optional, remove / comment the line if not wanted
%\extrainfo{additional information}                 % optional, remove / comment the line if not wanted
%\photo[64pt][0.4pt]{picture}                       % optional, remove / comment the line if not wanted; '64pt' is the height the picture must be resized to, 0.4pt is the thickness of the frame around it (put it to 0pt for no frame) and 'picture' is the name of the picture file
%\quote{Some quote}                                 % optional, remove / comment the line if not wanted

% to show numerical labels in the bibliography (default is to show no labels); only useful if you make citations in your resume
%\makeatletter
%\renewcommand*{\bibliographyitemlabel}{\@biblabel{\arabic{enumiv}}}
%\makeatother
%\renewcommand*{\bibliographyitemlabel}{[\arabic{enumiv}]}% CONSIDER REPLACING THE ABOVE BY THIS
\usepackage[document]{ragged2e}
% bibliography with mutiple entries
%\usepackage{multibib}
%\newcites{book,misc}{{Books},{Others}}
%----------------------------------------------------------------------------------
%            content
%----------------------------------------------------------------------------------
\begin{document}
%-----       letter       ---------------------------------------------------------
% recipient data
\recipient{ASML Recruitment team}{}%{Company, Inc.\\123 somestreet\\some city}
\date{\today}
\opening{Dear Sir or Madam,}
\closing{Yours faithfully,}
\enclosure[Attached]{curriculum vit\ae{}}          % use an optional argument to use a string other than "Enclosure", or redefine \enclname
\makelettertitle
\justify
I have recently learned about the new open vacancy related to the positions of \textbf{Software Design Engineer} and I am writing to state my interest on it. For completion, I attach my resume to this cover letter.

I am about to get a Ph.D. in Mathematics and Computer Science at the University of Calabria and I expect to finish by the end of October 2017. My Ph.D. programme has been mainly focused on scientific computation for massively parallel architectures for macroscopic natural phenomena simulation. I also have an MSc in Computer Science (\textit{cum laude}) from the University of Calabria, focused on parallel computing, theoretical computer science, AI and answer set programming.
As s Ph.D. graduate I wish to work for a leading international company, developing advanced and innovative products. 
After having challenged myself in different multicultural and professional backgrounds, by working on my MSc Thesis (for 8 months) at University of Plymouth, and by being accepted as Research Visiting student at  University of Edinburgh (10 months) and at University of Warwick (3 months), I am keen to work for a global, international-minded, company and be part of a proactive, dynamic and research-driven environment working on developing cutting-edge technology. 
I feel naturally driven to this type of work.
In addition to scientific computing, during my studies, I have also been interested in how the exascale simulation data could be interactively visualized and analyzed. This has been deepened with an enriching experience at the University of Edinburgh on the \textit{VELaSSCO} EU project.
My willingness and curiosity to learn, with my determination to pursue an objective with resilience, have always driven me through my studies and especially my Ph.D., where I have also learned to take ownership of my tasks while working and communicating effectively in a project-team with international partners. This mindset and the different
topics I have focused on so far, have given me the right adaptability and multidisciplinary background to fit in a company with an international outlook.

What attracts my of ASML is the possibility to work with top-class scientists and engineers to develop, design and build machines that are fundamental for large-scale production of microchips. Having the chance to work, as a passionate computer scientist, at the core of computing and for the largest supplier of chip-making equipment, tackling hard problems in complex fields such the ones that high-tech lithography and metrology pose excite me. 
Working on translating mathematical solutions produced by model experts to high-quality, well documented, fast and efficient software seems as a natural consequence of pursuing a Ph.D. in the field of High-Performance modeling and simulation.
The work on calibration and inaccuracies correction of the Metrology Software Department has an important role in making faster and smaller chips. The direct impact of the role on the quality of the product would push my enthusiasm more every day.
During the last 5 years, I have worked in interdisciplinary and international teams of mathematicians, physicists, experts in modeling and computer scientists. I have already achieved good academic results by producing HPC software from mathematical and/or prototypical specifications of solutions and consequently, I am also familiar with modeling methodologies and tools such as \texttt{Matlab}.
As an HPC programmer and a Ph.D. I combine strong theory with extensive experience in, among others, \texttt{C/C++}, \texttt{Unix}, build tools (\texttt{Make}, \texttt{\texttt{Cmake}}), versioning systems (\texttt{git}) and \texttt{GPGPU} (\texttt{CUDA}, \texttt{OpenCL}, \texttt{OpenMP4}). 

I believe that I can be a great fit for the position and for the team and I would be very grateful for the opportunity of an interview, where we can discuss about the job requirements and my ability to meet them.



\makeletterclosing

\end{document}


%% end of file `template.tex'.

\documentclass[a4paper,10pt]{article}

\usepackage[left=2.7cm,right=2.7cm,top=3.5cm,bottom=3cm]{geometry}
\usepackage{amsmath,amssymb,amsthm,amscd,amsfonts}
\usepackage{mathrsfs}
\usepackage{latexsym}
\usepackage[english]{babel}
\usepackage{eucal,eufrak}
\usepackage{verbatim}
\usepackage[lowtilde]{url}
\usepackage{color}
\usepackage{hyperref}
\usepackage{eurosym}
\usepackage{decorule}
\usepackage{swrule}
\title{Curriculum Vit$\ae$ et Studiorum}
\author{Davide Spataro}

\begin{document}

\maketitle

\tableofcontents

\newpage

\section{Personal information and research activity}

\subsection{Personal information}

\begin{table}[h]
\begin{tabular}{l l}
\textbf{Name and surname:} & Davide Spataro \\
\textbf{Address (Residential):} & Via Madonna della Scala 4/b ,89844 Nicotera
(VV), Italy \\
\textbf{Address (Residential):} & Via Belvedere 33075 Cordovado (PN), Italy
\\
\textbf{Address (UNICAL):} & Universit\`a della Calabria,High Performance
Computing Centre (HPCC),\\&  Ponte Pietro Bucci 22/B, 87036 Rende (CS),
Italy
\\
\textbf{Address (Warwick):} & Department of Computer Science -  University
of Warwick -\\
& Coventry CV4 7AL, UK.
\\
\textbf{E-mail:} &
\href{mailto:davide90.spataro@gmail.com}{davide90.spataro@gmail.com},
\href{mailto:davidespataro@davidespataro.it}{davidespataro@davidespataro.it}\\&
\href{mailto:d.spataro@mat.unical.it}{d.spataro@mat.unical.it}
\\
\textbf{Date and place of birth:} & 14 February 1990, Vibo Valentia (VV),
Italy \\
\textbf{Phone: }& +39 0963886097(landline), +39 3276324765(mobile),
+44 07958679367(mobile) \\
\textbf{Homepage: }& \url{http://www.davidespataro.it} \\
\end{tabular}
\end{table}

%SEPARATOR
\begin{center}
\mbox{}\swrulex{.5\textwidth}{.35pt}{2.5pt}
\end{center}
 
 
\subsection{Current position}
\begin{description}
 
 \item [\textbf{Ph.D Research Visiting Student}] at  
 \textit{Department of Computer Science}, \textit{University of Warwick} (Coventry, United Kingdom).
 
 \item [\textbf{Ph.D Student} and \textbf{Teaching Assistant}] at
  \textit{Department of Mathematics and Computer Science, University of Calabria} (Cosenza, Italy).
\end{description}

\subsection{Previous positions}
\begin{description}
 \item [\textbf{Ph.D Research Visiting Student}] at \textit{School of
  Engineering}, \textit{University of Edinburgh} (Edinburgh, United Kingdom).
  
\item [\textbf{Academic Tutor}] of \textit{Fundamentals of Computer Science} and
 \textit{Object-Oriented Programming} at the Department of Mathematics and
Computer Science, University of Calabria, Italy.
\end{description}

\subsection{Research interests}
Parallel Computing (GPGPU Computing, OpenMP, MPI, OpenACC, CUDA),
Parallel Computational Paradigms (Cellular Automata), Discrete Modeling and
Simulation, Scientific Visualization (Parallel Rendering, Computer Graphics,
Real-time rendering). Big-Data tools and Simulation (Hadoop ecosystem). 

\subsection{Research activity}
Starting from my MSc degree Thesis, I am collaborating with researchers from the
University of Calabria (Italy), Plymouth University (UK) and University of
Edinburgh (Scotland, UK) to studies on \textbf{Parallel Computing},
\textbf{Modeling and Simulations} in Computational Fluid Dynamics, and
\textbf{Scientific Visualization}.

In particular, in the \textbf{modeling and simulating field} I exploit the
computational power of Cellular Automata to model complex natural phenomena.

In the context of \textbf{Parallel Computing} my research focuses mainly on CUDA
(besides MPI, OpenMP, OpenACC) application to accelerate complex systems models
(e.g. simultaneous Cellular Automata models simulations).

I have worked on a parallel rendering tool as part of the VELaSSCO project
(EU 3.2M\euro $\:$ founded), which aims to be a  fast and scalable platform for
analysis of petascale numerical simulations.

Recently my interest is focused on fast the developing of a fast parallel tridiagonal system solver
and its integration within the library OPS.

\subsection{Short Biography}
Davide Spataro was born the $14^{th}$ February 1990 and grew up in Nicotera, a
small city in Southern Italy.
He attended the secondary school focusing on humanities and in 2008 he moved to Cosenza, city in
which he lives, studies and works collaborating with some researcher of the Department of Mathematics
and Computer Science on the modelling and simulation of complex systems and on
computational fluid-dynamics. He studied piano and music since he was eight for ten years, long enough to make him
addicted to classical and jazz music. In 2011, he obtained the Bachelor of Science in Computer Science
at the University of Calabria. and since 2014 he holds the Master of Science (summa cum laude) at
the University of Calabria. He is creative, and looking forward to earn a respectable scientists profile,
he is highly motivated and willing to learn and to work hard to achieve results. He also loves coffee.


%SEPARATOR
\begin{center}
\mbox{}\swrulex{.5\textwidth}{.35pt}{2.5pt}
\end{center}


\section{Education}

\subsection{University Education}

\paragraph{Master of Science in Computer Science.} 

I obtained the MSc degree in Computuer Science (\textbf{summa cum laude}) on
\textbf{July 2014} at the University of Calabria.\\
{\centering
\begin{tabular}{l l}
\textbf{Thesis title:}& \textsf{Accelerating the new SCIARA-fv3 numerical model
by different GPGPU strategies.}\\
\textbf{Thesis Supervisors:} & Prof. William Spataro, Prof. Donato D'Ambrosio\\
\end{tabular}
}

\noindent During the two-year degree the main courses I attended and exams taken are listed below: 

\noindent Data Warehousing and Data Mining, Knowledge Management, Modeling and Simulation, Numerical Approximation and Algorithms, 
Network and Computer Security, Parallel Algorithms and Distributed Systems, Theoretical Computer Science, Intelligent Systems, 
Cryptography and Coding Theory.


\paragraph{Bachelor of Science in Computer Science.} 

I obtained the BSc degree in Computer Science on \textbf{December 2011} at the
University of Calabria.\\ \hfill\\

{\centering
\begin{tabular}{l l}
\textbf{Thesis title:}& \textsf{B-finder a system for automatic detection of
buildings from aerophotogrammetries.} \\
\textbf{Thesis Supervisors:} & Prof. Pasquale Rullo, Prof. Salvatore Iiritano\\
\end{tabular}
}
\noindent During the three-year degree the main courses I attended and exams taken are listed below:

\noindent Analysis, Discrete Mathematics, Integral Calculus, Physics,
 Operations Research, Probability Theory and Statistics, Computer Architecture,
 Data bases, Object-Oriented Programming, Algorithms and Data Structures,
 Computer Graphics, Graphical Interfaces and Event-Oriented Programming,
 Artificial Intelligence, Formal Languages and Compilers, Operating Systems and
 Networks, Software Engineering, Web based Information systems.

%SEPARATOR
\begin{center}
\mbox{}\swrulex{.5\textwidth}{.35pt}{2.5pt}
\end{center}

\section{Research stays}
\subsection{National stays}

\begin{itemize}
\item \textbf{Intern} at  Exeura from 01/06/2011 to 21/12/2011 .
Working on my BSc Thesis on automated recognition of buildings using Computer Vision and Image Processing techniques.

\item \textbf{NESUS II Winter School}, on HPC and Exascale computing, Vibo Valentia, Italy, from 20/02/2017 to 23/02/2017.  
\end{itemize}


\subsection{International stays}


\begin{itemize}
 \item From 20/06/2017 to present I am a \textbf{research visiting student} at the
Department of Computer Science, University of Warwick(UK) under the
supervision of the Prof. \textit{Gihan Mudalige} working on 
developing a fast parallel tridiagonal solver for the OPS library.

 \item From 01/11/2015 to 30/07/2016 I was a  \textbf{research visiting student} at the
School of Engineering, University of Edinburgh (UK) under the
supervision of the Prof. \textit{Jin Ooi} working on Parallel Rendering of PETASCALE DEM
simulations.

\item From 01/03/2013 to 31/10/2013 I had a \textbf{research visit} at the
School of Computing and Mathematics, Plymouth University (UK) under the
supervision of the Prof. Davide Marocco.
During the visit period I worked on my  Thesis by applying GPGPU techniques to
the parallelization of the SCIARA-fv3 cellular automata model.
\end{itemize}


%SEPARATOR
\begin{center}
\mbox{}\swrulex{.5\textwidth}{.35pt}{2.5pt}
\end{center}

\section{Publications}



\begin{itemize}

\item Donato D'Ambrosio, Alessio De Rango, Davide Spataro, Rocco Rongo and William Spataro. \textbf{Applications of the OpenCAL Scientific Library in the context of CFD: Applications to Debris Flows}, \emph{Proceedings of The 2017 International
  International Conference on Networking, Sensing and Control (ICNSC),
  May 16-18, 2017, Calabria, Italy} 

  \item Davide Spataro, Paola Arcuri, Donato D'Ambrosio, Alessio De Rango, Alice Mari and William Spataro, \textbf{A Tracking Algorithm for Particle-like Moving Objects}, \emph{Proceedings of The 2017 International
  Conference on Parallel, Distributed and Network-Based Processing (PDP),
  March 6-8 2016, St. Petersburg, Russia
  }
  \item Rahmat Hidayat, Davide Spataro, Elisa De Giorgio, William Spataro,
  Donato D'Ambrosio, \textbf{Multi-Agent System with Multiple Group Modelling
  for Bird Flocking on GPU}, \emph{Proceedings of The 2016 International
  Conference on Parallel, Distributed and Network-Based Processing (PDP),
  February 17-19 2016, Crete, Greece}

\item Alessio De Rango, Mauizio Macr\'i, Davide Spataro, Donato D'Ambrosio and
  William Spataro, \textbf{Efficient Lava Flows Simulations with OpenCL: A
  preliminary application for Civil Defence Purposes}, \emph{Proceedings of
  The $10^{th}$ International Conference on P2P, Parallel, Grid, Cloud and
  Internet Computing, November 4-6, 2015, Krakow, Poland}
  
	\item Filippone G., Spataro W., D'Ambrosio D., Spataro D., 
	Marocco D.,Trunfio G.A., \textbf{CUDA Dynamic Active Thread List Strategy 
	to Accelerate Debris Flow Simulations} \emph{Proceedings of The 2015 International Conference on Parallel, 
	Distributed and Network-Based Processing (PDP)}, Turku, Finland, pp 330-338.
	
	\item Spataro D., D'Ambrosio D., Filippone G., Spataro W., \textbf{The new
	SCIARA-fv3 numerical model and acceleration by GPGPU strategies}
	\emph{International Journal of High Performance and Applications},
	\texttt{doi:10.1177/1094342015584520}.
	
	\item Spataro W., D'Ambrosio D., Filippone G.,Spataro D., G.
    Iovine, D. Marocco, \textbf{Lava flow modeling by the SCIARA-fv3
    parallel numerical model}, \emph{Proceedings of The 2014 International
    Conference on Parallel, Distributed and Network-Based Processing (PDP)}, Turin, Italy, Feb. 12-14,
    2014, pp 330-338.
    
    \item G. Filippone, R. Parise, D. Spataro,
	D. D'Ambrosio, R. Rongo, and W. Spataro, \textbf{Evolutionary
	applications to Cellular Automata models for volcano risk mitigation},
	\emph{Proceedings of The 2014 International Conference on Workshop on Artificial Life and
	Evolutionary Computation (WIVACE)}, May 14-15 2014, Vietri sul Mare, Salerno,
	Italy.
   

\end{itemize}

    %SEPARATOR
\begin{center}
\mbox{}\swrulex{.5\textwidth}{.35pt}{2.5pt}
\end{center}

\section{Projects}

\subsection{Academic}
 \paragraph{SuSy Project - Predictive Survey}
 System Definition of a predictive
 model based on neural networks and genetic algorithms for the classification of
 statistical surveys responses. (2014/2015)

\paragraph{OpenCAL - CUDA Complex Cellular Automata library} 
OpenCAL (Open Cellular Automata Library) is a parallel multi-platform library
for Complex Cellular Automata (CCA) (2015).
\hfill \\
\texttt{- \noindent Cellular automata (complex and classic), CUDA, GPGPU, MPI, OPENGL,
OPENMP.}
 
 \paragraph{cuCompact - CUDA Stream Compaction} 
A CUDA powered module for efficient and fast generic stream compaction on GPU,
using intra-warp ballotting intrinsic function. 
\hfill \\
\texttt{- 
\noindent Cellular automata (complex and classic), CUDA, GPGPU.}

 \paragraph{ACIADDRI - Multi Agents CUDA/OPENGL Bird Flock System} 
Multi-Agent multiple group bird flocking modelling system open GPU using
CUDA/OpenGL interoperability.
\hfill \\
\texttt{- 
\noindent Cellular automata (complex and classic), CUDA, GPGPU.}


\paragraph{CuCCAl - CUDA Complex Cellular Automata library} 
A CUDA powered library “CuCCAl”, acronym for “Cuda complex cellular
automata library”, capable of hiding all the complexity behind the GPUGPU
Complex Cellular Automaton programming process.
\hfill \\
\texttt{- 
\noindent Cellular automata (complex and classic), CUDA, GPGPU.}

 \subsection{BS and MS degree}

\paragraph{Robocode Competition} 
I developed, in a team of three, a robocode robot using the Robocode framework
within the context of the final exam and competition of the Intelligent System's
course during Master's degree. My robot wins the competition.

\noindent During this project i worked in a team of three people mainly
appliying concept and tecniques from :
\noindent Artificial intelligence and data mining and gathering.
 

 \paragraph{A 3D videogame: Big-Enthity } 

I developed, in a team of two, a 3D videogame alongside a level editor using
java and jMonkey 3D engine and jBox2D physics engine.
It was developed as project for the course of Advanced Programming during my
bachelor's degree.

\noindent During this project i worked in a team of three people mainly
appliying concept and tecniques from:
\noindent 3D programming, event programming, XML domain modelling, advanced
data structures and algorithms.

 \paragraph{Parallel linear system solver} 

I developed, using MPI and OpenMP a  parallel program for solving linear system
of equations. It was part of the exam of Parallel algorithms and
Distribuited System. Very good result in terms of efficiency and speedups were
achived.

\end{document}

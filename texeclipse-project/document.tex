\documentclass[a4paper,10pt]{article}

\usepackage[left=2.7cm,right=2.7cm,top=3.5cm,bottom=3cm]{geometry}
\usepackage{amsmath,amssymb,amsthm,amscd,amsfonts}
\usepackage{mathrsfs}
\usepackage{latexsym}
\usepackage[italian]{babel}
\usepackage{eucal,eufrak}
\usepackage{verbatim}
\usepackage[lowtilde]{url}
\usepackage{color}
\usepackage{hyperref}

\title{Curriculum Vit$\ae$ et Studiorum}
\author{Davide Spataro}

\begin{document}

\maketitle

\tableofcontents

\newpage

\section{Personal information and research activity}

\subsection{Personal information}

\begin{table}[h]
\begin{tabular}{l l}
\textbf{Name and surname:} & Davide Spataro \\
\textbf{Address :} & Via Madonna della Scala 4/b ,89844 Nicotera (VV), Italy
\\
\textbf{Address :} & Universit\`a della Calabria, HPCC, Ponte Pietro Bucci
22/B, 87036 Rende (CS), Italy \\
\textbf{E-mail:} &
\href{mailto:davide90.spataro@gmail.com} {davide90.spataro@gmail.com},
\href{mailto:davidespataro@davidespataro.it.com}
{davidespataro@davidespataro.it}\\
\textbf{Date and place of birth:} & 14 February 1990, Vibo Valentia (VV),
Italy \\
\textbf{Phone: }& +39 0963886097(home), +39 276324765(mobile),
+44 07449743894(mobile) \\
\textbf{Homepage: }& \url{http://www.davidespataro.it} \\
\end{tabular}
\end{table}

\subsection{Current position}
Tutor of  Fundamentals of Computer Science and Object-Oriented Programming at 
the Department of Mathematics and Computer Science, University of Calabria,
Italy.

\subsection{Research interests}
Parallel Computing (GPGPU Computing, OpenMP, MPI, OpenACC, CUDA),
Parallel Computational Paradigms (Cellular Automata), Discrete Modeling and
Simulation, Scientific Visualization (Computer Graphics, Real-time rendering), Mobile developing (Android).

\subsection{Research activity}
Starting from my MSc degree Thesis, I am collaborating with researchers from the
University of Calabria (Italy) and from the Plymouth University (UK) to studies
on \textbf{Parallel Computing}, \textbf{Modeling and
Simulations} in Computational Fluid Dinamics, and \textbf{Scientific Visualization}.

In particular, in the \textbf{modelling and simulating field} I exploit the
computational power of Cellular Automata to model complex natural phenomena.

In the context of \textbf{Parallel Computing} my research focuses mainly on CUDA
(besides MPI, OpenMP, OpenACC) application to accelerate complex systems models
(e.g. simultaneous Cellular Automata models simulations).

\subsection{Short Biography}
I come from Nicotera, Calabria (Southern Italy). In 2008 I moved to Cosenza,
city in which actually i live, study and work.

In 2011, I obtained my Bachelor of Science in Computer Science at the University
of Calabria.

In 2014, I completed a Master of Science (\textbf{summa cum laude}) at the
University of Calabria.


\section{Education}

\subsection{University education}

\paragraph{Master of Science in Computer Science.} 

I obtained the MSc degree in Computuer Science (\textbf{summa cum laude}) on
\textbf{July 2014} at the University of Calabria.\\
{\centering
\begin{tabular}{l l}
\textbf{Thesis title:}&Accelerating the new SCIARA-fv3 numerical model by
different GPGPU strategies.\\
\textbf{Thesis Supervisors:} & Prof. William Spataro, Prof. Donato D'Ambrosio\\
\end{tabular}
}

\noindent During the two-year degree the main courses I attended and exams taken are listed below: 

\noindent Data Warehousing and Data Mining, Knowledge Management, Modeling and Simulation, Numerical Approximation and Algorithms, 
Network and Computer Security, Parallel Algorithms and Distributed Systems, Theoretical Computer Science, Intelligent Systems, 
Cryptography and Coding Theory.

\paragraph{Bachelor of Science in Computer Science.} 

I obtained the BSc degree in Computer Science on \textbf{December 2011} at the
University of Calabria.\\

{\centering
\begin{tabular}{l l}
\textbf{Thesis title:}&B-finder a system for automatic detection of buildings from aerophotogrammetries. \\
\textbf{Thesis Supervisors:} & Prof. Pasquale Rullo, Prof. Salvatore Iiritano\\
\end{tabular}
}
\noindent During the three-year degree the main courses I attended and exams taken are listed below:

\noindent Analysis, Discrete Mathematics, Integral Calculus, Physics,
 Operations Research, Probability Theory and Statistics, Computer Architecture,
 Data bases, Object-Oriented Programming, Algorithms and Data Structures,
 Computer Graphics, Graphical Interfaces and Event-Oriented Programming,
 Artificial Intelligence, Formal Languages and Compilers, Operating Systems and
 Networks, Software Engineering, Web based Information systems.

\section{Research stays}
\subsection{National stays}

\begin{itemize}
\item From 01/06/2011 to 21/12/2011 I had a \textbf{Stage} at Exeura.
During the stage period I worked on the develompment of my BSc Thesis
improving my computer vision, MATLAB, and image processing skills.
\end{itemize}

\subsection{International stays}


\begin{itemize}
\item From 01/03/2013 to 31/10/2013 I had a \textbf{research visit} at the
School of Computing and Mathematics, Plymouth University (UK) under the
supervision of the Prof. Davide Marocco.
During the visit period I worked on my  Thesis by applying GPGPU techniques to
the parallelization of the SCIARA-fv3 cellular automata model.
\end{itemize}


\section{Publications}



\begin{itemize}
%
\item Spataro D., D'Ambrosio D., Filippone G., Spataro W., \textbf{Implementation of the SCIARA-fv3
    parallel numerical model for lava flow simulation by different GPGPU
    strategies} \emph{International Journal of High Performance and
    Applications}. Accepted.
\item Spataro W., D'Ambrosio D., Filippone G.,Spataro D., G.
    Iovine, D. Marocco, \textbf{Lava flow modeling by the SCIARA-fv3
    parallel numerical model}, \emph{Proceedings of The 2014 International
    Conference on Parallel, Distributed and Network-Based Processing (PDP)}, Turin, Italy, Feb. 12-14,
    2014, pp 330-338.
    \item G. Filippone, R. Parise, D. Spataro,
D. D'Ambrosio, R. Rongo, and W. Spataro, \textbf{Evolutionary
applications to Cellular Automata models for volcano risk mitigation},
\emph{Proceedings of The 2014 International Conference on Workshop on Artificial Life and
Evolutionary Computation (WIVACE)}, May 14-15 2014, Vietri sul Mare, Salerno,
Italy.
   

\end{itemize}

    


\end{document}